\chapter{Feed Forward Neural Networks}

\section{Étape de la construction du modèle}

Pour construire un modèle de réseau de neurones, il est nécessaire de définir l'architecture du réseau, les entrées, les sorties et les fonctions d'activation, ainsi que les fonctions de coût. 
Pour ce faire, l'étude préalable a servi de base pour sélectionner les paramètres de manière optimale.

\section{Embeddings}

Les \textit{embeddings} sont une technique couramment utilisée pour représenter des données de manière dense et continue dans des espaces de grande dimension. 
Ils sont particulièrement utiles pour traiter des données textuelles et des catégories discrètes.
Pour réaliser une classification multilabel, il a été décidé d'utiliser deux types d'\textit{embeddings} différents : \textit{GloVe} et \textit{FastText}.

\subsubsection{GloVe}

\textit{GloVe} (Global Vectors for Word Representation) est un algorithme de vectorisation de mots qui représente les mots sous forme de vecteurs de dimension prédéfinie. 
Il utilise des statistiques globales de cooccurrence des mots pour capturer des relations contextuelles subtiles, ce qui le rend particulièrement adapté à des tâches comme la classification de texte.

\subsubsection{FastText}

\textit{FastText} est un modèle d'\textit{embeddings} développé par \textit{Facebook AI Research (FAIR)} qui améliore les techniques de vectorisation de mots en prenant en compte les sous-mots. 
Contrairement aux méthodes traditionnelles qui représentent les mots entiers comme des unités indépendantes, \textit{FastText} décompose chaque mot en n-grammes de caractères, ce qui permet de capturer les relations morphologiques entre les mots.


\section{Architecture du modèle}

Le modèle de réseau de neurones \textit{feedforward (FNN)} construit est composé de plusieurs couches entièrement connectées. Voici les détails de l'architecture du modèle :
\begin{itemize}
    \item \textbf{Couche d'entrée}: Reçoit les données d'entrée de dimension \textbf{200}.
    \item \textbf{Couche cachée}: Une couche entièrement connectée de dimension \textit{hidden\_dim} suivie d'une normalisation par lots (\texttt{BatchNorm1d}).
    \item \textbf{Fonction d'activation}: \textit{ReLU (Rectified Linear Unit)} après la couche cachée.
    \item \textbf{Couche de sortie}: Une couche entièrement connectée de dimension \textbf{6} suivie d'une fonction d'activation sigmoïde pour la classification multilabel de chaque catégorie.
\end{itemize}

\subsubsection{Fonction de coût}

Pour entraîner le modèle, la fonction de coût utilisée est la vraisemblance négative logarithmique (\textit{BCEWithLogitsLoss}) qui est couramment utilisée pour les tâches de classification multilabel.
De plus, la pondération des classes a été utilisée pour compenser le déséquilibre des classes dans le jeu de données.
\subsubsection{Optimisation}

Pour optimiser le modèle, l'algorithme d'optimisation employé est \textit{Adam} (AdamOptimizer), qui est une méthode d'optimisation stochastique basée sur l'estimation adaptative des moments.

\section{Création des modèles}

Afin de trouver la meilleure architecture du modèle, il a été décidé de tester plusieurs configurations de réseaux de neurones en faisant varier trois paramètres :

\begin{itemize}
    \item \textbf{Nombre de neurones dans la couche cachée} : \textbf{32},\textbf{64},\textbf{128}
    \item \textbf{Type d'embedding utilisé} : \textit{GloVe} et \textit{FastText}
    \item \textbf{Type de prétraitement des données textuelles} : Différentes méthodes de prétraitement des données textuelles ont été explorées, telles que la normalisation, la suppression des stop-words et la lemmatisation, afin de trouver la combinaison optimale pour cette tâche de classification \textit{multilabel}.
\end{itemize}


\section{Résultats}

Voici les scores de précision, de rappel et de F1-Score par label, d'un modèle avec une couche cachée de 32 neurones, utilisant les \textit{embeddings} de \textit{FastText} et aucun prétraitement des données textuelles.

\begin{table}[ht]
    \centering
    \caption{Scores de Précision, Rappel, et F1-Score par Label, d'un modèle de réseau de neurones feedforward \textit{FastText} Optimisé pour chaque label, sur le jeu de test}    \begin{tabular}{lcccc}
    \hline
    \textbf{Label} & \textbf{Précision} & \textbf{Rappel} & \textbf{F1-Score} & \textbf{Support} \\ \hline
    toxic          & 0.64               & 0.50            & 0.56              & 6090             \\
    severe\_toxic  & 0.27               & 0.22            & 0.24              & 367              \\
    obscene        & 0.71               & 0.46            & 0.56              & 3691             \\
    threat         & 0.50               & 0.07            & 0.12              & 211              \\
    insult         & 0.66               & 0.36            & 0.47              & 3427             \\
    identity\_hate & 0.50               & 0.02            & 0.04              & 712              \\
    overall\_non\_toxic & 0.95           & 0.97            & 0.96              & 57735            \\\hline
    macro avg      & 0.60              & 0.37            & 0.42              & 72233            \\
    weighted avg   & 0.89               & 0.86            & 0.87              & 72233            \\ \hline
    \end{tabular}
    \label{tab:scores}
\end{table}

Après avoir entraîné et évalué plusieurs modèles de réseaux de neurones, il a été constaté que le modèle avec une couche cachée de 64 neurones, utilisant les \textit{embeddings} de \textit{GloVe} et un prétraitement des données textuelles comprenant la suppression des caractères spéciaux, la mise en minuscules et la suppression des duplications, a obtenu les meilleurs résultats.
\begin{table}[ht]
    \centering
    \caption{Scores de Précision, Rappel, et F1-Score par Label, d'un modèle de réseau de neurones feedforward \textit{GloVe} Optimisé pour chaque label, sur le jeu de test}    \begin{tabular}{lcccc}
    \hline
    \textbf{Label} & \textbf{Précision} & \textbf{Rappel} & \textbf{F1-Score} & \textbf{Support} \\ \hline
    toxic          & 0.55               & 0.72            & 0.62              & 6090             \\
    severe\_toxic  & 0.33               & 0.47            & 0.38              & 367              \\
    obscene        & 0.62               & 0.63            & 0.62              & 3691             \\
    threat         & 0.52               & 0.42            & 0.46              & 211              \\
    insult         & 0.59               & 0.61            & 0.60              & 3427             \\
    identity\_hate & 0.49               & 0.37            & 0.42              & 712              \\
    overall\_non\_toxic & 0.97           & 0.94            & 0.95              & 57735            \\\hline
    macro avg      & 0.58              & 0.59            & 0.58              & 72233            \\
    weighted avg   & 0.89               & 0.88            & 0.88              & 72233            \\ \hline
    \end{tabular}
    \label{tab:scores}
\end{table}