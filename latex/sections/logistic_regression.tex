\chapter{Régression logistique}

\subsubsection*{Présentation du modèle}

Ce modèle est un bon point de départ pour la classification de texte, car il est simple, rapide et efficace pour les tâches de classification binaire.
Pour créé un modèle qui répond à notre problème de classification multi-label, nous allons utiliser la méthode MulitOutputClassifier de la librairie scikit-learn.
Cette méthode permet de transformer un problème de classification multi-label en plusieurs problèmes de classification binaire.
Nous allons donc entraîner un modèle de régression logistique pour chaque label de toxicité du jeu de données puis combiner les prédictions pour obtenir un modèle multi-label.

\subsubsection*{Prétraitement des données}

Pour prétraiter les données, nous avons utilisé la méthode TfidfVectorizer de la librairie scikit-learn.
Cette méthode permet de transformer les commentaires en vecteurs de fréquence de termes-inverse de document (TF-IDF) qui nous permettent de représenter les commentaires sous forme de vecteurs numériques.
Nous avons également utilisé la méthode de prétraitement de texte de la librairie NLTK pour nettoyer les commentaires en supprimant les caractères spéciaux, les chiffres et les mots vides.

\subsubsection*{Recherche des hyperparamètres}
Une fois le modele établie et les données prétraitées, nous allons chercher les meilleurs hyperparamètres pour notre modèle en utilisant la méthode bayésienne d'optimisation des hyperparamètres à l'aide de la libraire optuna.
Ainsi nous allons pouvoir obtenir un modèle de régression logistique qui prédit la toxicité des commentaires de manière efficace et équitable.
Voici les hyperparamètres de notre modèle ainsi que ses performances sur le jeu de données de test :

\begin{table}[h]
\centering
\begin{tabular}{|l|l|}
\hline
\textbf{Hyperparamètre} & \textbf{Valeur} \\ \hline
C                       & 4.24             \\ \hline
penalty                 & l1            \\ \hline
solver                  & liblinear           \\ \hline
max\_iter               & 1000           \\ \hline
\end{tabular}
\caption{Hyperparamètres du modèle de régression logistique}
\end{table}

\subsubsection*{Performances}

Voici les performances du modèle de régression logistique sur le jeu de données de test :

\begin{table}[h]
    \centering
    \begin{tabular}{lcccc}
        \hline
        \textbf{Classe} & \textbf{Précision} & \textbf{Rappel} & \textbf{F1-Score} & \textbf{Support} \\
        \hline
        Toxic          & 0.64 & 0.72 & 0.67 & 6090 \\
        Severe Toxic   & 0.37 & 0.31 & 0.34 & 367 \\
        Obscene        & 0.74 & 0.62 & 0.68 & 3691 \\
        Threat         & 0.54 & 0.26 & 0.35 & 211 \\
        Insult         & 0.72 & 0.53 & 0.61 & 3427 \\
        Identity Hate  & 0.67 & 0.26 & 0.37 & 712 \\
        \hline
        \textbf{Micro Avg} & 0.67 & 0.61 & 0.64 & 14498 \\
        \textbf{Macro Avg} & 0.61 & 0.45 & 0.50 & 14498 \\
        \textbf{Weighted Avg} & 0.68 & 0.61 & 0.63 & 14498 \\
        \textbf{Samples Avg} & 0.06 & 0.06 & 0.06 & 14498 \\
        \hline
    \end{tabular}
    \caption{Rapport de classification des différentes classes}
    \label{tab:classification_report}
\end{table}
