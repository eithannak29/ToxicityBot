\pagenumbering{arabic}

\chapter{Introduction}

Ce projet vise à développer des modèles de traitement du language naturel dédiés à la modération en temps réel des chats sur une plateforme de contenu. L'objectif principal est d'assurer la modération des messages des utilisateurs afin de prévenir la diffusion de contenus inappropriés, tels que les insultes, les spams et autres formes de communications indésirables.
\subsubsection*{Jeu de données} 
Le jeu de données choisi pour ce projet est le \textbf{Jigsaw Toxic Comment Classification}.
Ce jeu de données recense un grand nombre de commentaires anglais, provenant de \textit{Wikipedia}, labellisés par des humains en fonction de leur toxicité.
Ce dernier provient d'une compétition \textit{Kaggle} et peut être obtenu à partir du lien suivant: \href{https://www.kaggle.com/competitions/jigsaw-toxic-comment-classification-challenge/data}{jigswaw toxic comment classification challenge}.

\subsubsection*{Objectif}
Plusieurs modèles de \textit{machine learning} et de \textit{deep learning} seront développés et entraînés dans le but de prédire la pertinence et l'acceptabilité des messages.
Une fois ces modèles développés, nous procéderons à une évaluation rigoureuse de leurs performances par comparaison mutuelle. 
La dernière étape de ce projet consistera en la création d'une preuve de concept sous la forme d'une interface web. 

Cette interface permettra de modérer en direct les messages échangés sur un chat de \textit{live streaming}, démontrant ainsi l'applicabilité pratique de nos recherches.
