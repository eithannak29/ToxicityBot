\usepackage [left=2.5cm, right=3.5cm, top=3cm, bottom=3cm]{geometry}
\setlength{\headheight}{14.49998pt}%\geometry{showframe}

\usepackage{fancyhdr}
\pagestyle{fancy}
\usepackage{tabularx}


\usepackage[T1]{fontenc}      % for T1 font encoding
\usepackage[english]{babel}
\usepackage{fancyhdr}

% Configuration du package fancyhdr
\pagestyle{fancy}
\fancyhf{} % nettoie l'en-tête et le pied de page par défaut
\fancyhead[L]{\nouppercase{\leftmark}} % En-tête gauche : nom du chapitre
\fancyfoot[L]{} % Pied de page gauche : auteur
\fancyfoot[C]{NLP - Chat Moderation} % Pied de page centre : titre du document
\fancyfoot[R]{\thepage} % Pied de page droit : numéro de page

% Pour s'assurer que le en-tête et pied de page apparaissent sur la première page d'un chapitre
\fancypagestyle{plain}{
\fancyhf{}
  \fancyhead[L]{\nouppercase{\leftmark}}
  \fancyfoot[L]{}
  \fancyfoot[C]{NLP - Chat Moderation}
  \fancyfoot[R]{\thepage}
}

\usepackage[utf8]{inputenc}
\usepackage{graphicx}
\usepackage{titling}
\usepackage{datetime}
\usepackage{tikz}

\usepackage{appendix}

\makeatletter
\renewcommand\Huge{\@setfontsize\huge{18}{18}}
\renewcommand\huge{\@setfontsize\huge{18}{18}}
\renewcommand\Large{\@setfontsize\large{15}{15}}
\renewcommand\large{\@setfontsize\large{15}{15}}

\makeatother
\linespread{1.2}

% \usepackage[top=3cm, bottom=3cm, headheight=15pt, headsep=10mm, footskip=10mm]{geometry}

\usepackage{hyperref}
\hypersetup{
    colorlinks,
    citecolor=blue,
    filecolor=blue,
    linkcolor=black,
    urlcolor=blue
}

\usepackage{glossaries}
\makeglossaries
\newacronym{ai}{AI}{Artificial Intelligence}
\newacronym{ml}{ML}{Machine Learning}


\usepackage{titlesec}

% Configuration de titlesec pour les titres de chapitre
\titleformat{\chapter}[block]
  {\normalfont\bfseries\Huge}{\thechapter.}{1em}{\normalfont\bfseries}
\titlespacing*{\chapter}{0pt}{-30pt}{5pt}